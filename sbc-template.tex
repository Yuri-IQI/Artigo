\documentclass[12pt]{article}

\usepackage{etoolbox}
\usepackage{sbc-template}
\usepackage{graphicx,url}
\usepackage[utf8]{inputenc}
\usepackage[brazil]{babel}
\usepackage[none]{hyphenat}
\usepackage{amsmath}
\usepackage{amssymb} 
\usepackage{microtype}
\usepackage{caption}
\usepackage{setspace}
\usepackage{lipsum}
\usepackage{float}

\captionsetup{
    labelsep=period,
    font=small,
    justification=centering
}

\setlength{\intextsep}{10pt}
\setlength{\floatsep}{12pt}

\makeatletter
\patchcmd{\@startsection}
{\@afterindentfalse}
{\@afterindenttrue}
{}{}
\makeatother
     
\sloppy

\title{Cobertura e Acessibilidade: Aplicação de P-Medianas e PLMC em um Recorte Urbano}

\author{Anthony França, Antônio Neto, Enzo Santana, Franck Vasconcelos,\\
Murilo Mota, Rafael Gonçalves, Rene Marinho}

\address{Universidade Tiradentes (UNIT)\\}
\date{2025}

\begin{document} 

\maketitle
     
\begin{resumo}
Este estudo apresenta um panorama das principais formulações dos Problemas de Localização de Instalações (Facility Location Problems — FLPs) e suas aplicações em contextos urbanos. Apresenta-se também um cenário ilustrativo que compara duas formulações clássicas: o Problema das P-Medianas e o Problema de Localização de Máxima Cobertura (PLMC). Descrevem-se as metodologias empregadas para cada variante e são discutidas as implicações práticas e as limitações de uma análise exploratória baseada em dados abertos e uma aplicação demonstrativa, destacando como diferentes abordagens servem a objetivos distintos, reduzir deslocamentos médios ou ampliar a cobertura territorial e como esses problemas podem ser modelados em solvers simples.
\end{resumo}

\section{Introdução}

O posicionamento ideal de instalações é crucial para a provisão eficiente de serviços, sejam eles de natureza comercial ou pública, assim como para a integração regional por meio de infraestruturas físicas, como pontes, aeroportos e terminais rodoviários. No âmbito da Pesquisa Operacional, esse desafio é tratado pelo Problema de Localização de Instalações (PLI), cujo objetivo é selecionar locais de instalação que maximizem o atendimento à demanda e a acessibilidade dos usuários. Tais decisões devem considerar critérios diversos, como a maximização da cobertura da demanda, a minimização dos custos operacionais e a garantia de níveis mínimos de atendimento para determinados grupos de clientes (Sousa Filho et al., 2012; Rushton, 1979).

Várias formulações matemáticas têm sido propostas para atender a esses objetivos, destacando-se, entre elas, o Problema de Localização de Máxima Cobertura (PLMC) e o Problema das P-Medianas. Esses modelos dependem de fatores como a distribuição geográfica da população e os tempos de deslocamento para determinar soluções viáveis e eficientes. De fato, estudos recentes apontam que esses critérios de localização buscam garantir a acessibilidade, ponderando simultaneamente a demanda por serviços e as distâncias de viagem (Kuo e Kung, 2025).

Entre os cenários de aplicação dos PLI, destaca-se a relação entre instalações e clientes. Quando a instalação a ser aberta é escolhida dentre um conjunto de locais potenciais e os clientes são alocados a essas instalações visando minimizar o custo total de atendimento à demanda, obtém-se a formulação conhecida como Problema de Localização de Instalações sem Limitação de Capacidade (Uncapacitated Facility Location Problem - UFLP). Por outro lado, quando as demandas dos clientes devem ser atendidas exclusivamente por instalações com capacidade limitada, sem possibilidade de fracionar a demanda entre diferentes locais, o problema é classificado como o Problema de Localização de Instalações Capacitado de Fonte Única (Single-Source Capacitated Facility Location Problem - CFLP).

Tanto o UFLP quanto o CFLP baseiam-se na minimização dos custos de transporte entre clientes e instalações, desconsiderando fatores subjetivos, como a preferência dos usuários por determinada instalação (o que pode influenciar sua propensão a utilizá-la). Nesse contexto, esses problemas assumem caráter NP-difícil, uma vez que é necessário considerar numerosas combinações de alocação de clientes a instalações (Kang et al., 2023; Büsing et al., 2024).

Grande parte das formulações propostas adota cenários determinísticos, nos quais variáveis de demanda e tempo de viagem são tratadas como valores fixos. No entanto, dada a natureza NP-difícil desses problemas, surgem limitações inerentes na busca por soluções ótimas. Mesmo os modelos clássicos, que fornecem descrições generalistas adequadas, apresentam restrições significativas em contextos de alta incerteza ou dinamicidade, que poderiam refletir de forma mais realista a complexidade dos cenários reais. Incertezas e aspectos dinâmicos podem ser incorporados aos modelos de localização por meio de ferramentas como programação estocástica e análise de cenários (Owen e Daskin, 1998).

A seguir, são apresentadas considerações sobre a motivação e os objetivos deste trabalho. Em seguida, traçaremos um panorama dos principais estudos na área de localização de instalações, enfatizando os trabalhos mencionados (especialmente a metodologia de análise utilizada por Rushton e a revisão de Owen e Daskin). Posteriormente, na seção de Fundamentação Teórica, descreveremos as formulações adotadas para os problemas estudados, e, por fim, nas seções de Metodologia e Resultados, detalharemos os procedimentos adotados na análise do cenário.

\subsection{Importância do tema}

Os problemas de localização de instalações constituem um campo consolidado de pesquisa em otimização e planejamento espacial, com aplicações que abrangem desde diagnósticos da infraestrutura existente até modelos destinados ao suporte de decisões estratégicas para alocação futura de recursos. Apesar das limitações inerentes à simulação de cenários reais, especialmente quando os dados disponíveis são incompletos, heterogêneos ou apresentam incertezas, esse conjunto teórico oferece instrumentos valiosos para examinar a relação entre a oferta de serviços e os padrões territoriais de demanda.

No contexto deste trabalho, adotou-se uma abordagem simplificada e exploratória, fundamentada na extração e no tratamento de dados abertos provenientes do OpenStreetMap (OSM). Tais procedimentos permitiram gerar representações cartográficas e indicadores iniciais de acessibilidade e cobertura, capazes de evidenciar desigualdades ou concentrações espaciais relevantes. Assim, a relevância prática deste exercício reside na demonstração de como dados geoespaciais acessíveis podem ser transformados em evidências interpretáveis, contribuindo para reflexões preliminares no âmbito do planejamento urbano e da análise territorial.

\subsection{Objetivos}

O objetivo geral deste estudo foi conduzir uma análise exploratória, baseada em métodos clássicos de Localização de Instalações, para examinar a distribuição espacial de farmácias em bairros da região central de Aracaju, utilizando dados geoespaciais obtidos do OpenStreetMap por meio das plataformas uMap e Overpass.

De forma mais específica, buscou-se:

\begin{itemize}
    \item Obter, selecionar e preparar dados geoespaciais do OSM referentes a nós de demanda e de serviço na área estudada;
    \item Criar uma ferramenta de resolução demonstrativa para simular o cenário.
    \item Ilustrar, de maneira conceitual e aplicada, o comportamento esperado de duas formulações clássicas da literatura: o Problema das P-Medianas e o Problema de Localização de Máxima Cobertura (PLMC).
\end{itemize}

Ressalta-se que o escopo deste trabalho não contemplou a execução de otimizações numéricas completas, nem a modelagem probabilística da demanda. Optou-se por uma abordagem descritiva e pragmática, compatível com o caráter exploratório da investigação e com o nível de detalhamento dos dados disponíveis.

\section{Trabalhos Relacionados}

Segundo Owen e Daskin (1998), a teoria da localização de instalações teve início em 1909 com o trabalho pioneiro de Alfred Weber, que buscava posicionar um depósito de modo a minimizar a distância total até os clientes. Rushton (1978) descreve esses primeiros esforços como essencialmente analíticos e restritos a formulações matemáticas básicas e representações gráficas rudimentares, com aplicação prática limitada. Somente na década de 1960, com o advento da computação, pesquisadores puderam propor métodos aplicáveis a cenários reais de localização. Mesmo assim, as formulações clássicas mantiveram-se estáticas e determinísticas, considerando parâmetros de demanda e tempos de deslocamento como dados fixos, o que restringe sua flexibilidade diante de condições variáveis.

Estudos mais recentes têm buscado expandir essas abordagens tradicionais, incorporando conceitos de modularidade e dinamicidade. Alarcon-Gerbier e Buscher (2022) ressaltam que a modularidade, viabilizada por meio de unidades de produção realocáveis ou expansíveis e instalações móveis, vem recebendo atenção crescente, com aplicações emergentes em diversos setores. De modo similar, Kang et al. (2023) propuseram um modelo de localização de serviços que considera as preferências dos clientes pelas instalações, além de levar em conta explicitamente as capacidades das unidades, evidenciando a complexidade adicional introduzida por esses fatores. Nesses modelos modernos, pesquisadores ponderam não apenas os custos de transporte e a cobertura da demanda, mas também escolhas comportamentais dos usuários, ampliando os requisitos de viabilidade do sistema.

Em “Uma arquitetura e ferramentas para problemas de localização de facilidades no setor público”, Filho et al. (2010) apresentam uma arquitetura modular para aplicações web voltadas à resolução de FLPs. A proposta organiza o sistema em dois módulos principais: um módulo de otimização, responsável pela execução dos algoritmos de solução, e um módulo de interface, encarregado da interação com o usuário e da visualização dos resultados em ambiente geográfico. Essa separação funcional favorece a escalabilidade e a flexibilidade da aplicação, permitindo sua adaptação a diferentes cenários de planejamento urbano e gestão pública, em conformidade com práticas recomendadas de desenvolvimento de sistemas na área.

As primeiras proposições forneceram uma base conceitual sólida, porém restrita a cenários ideais de operação e atendimentos simplificados. Com o avanço da área, passaram a ser considerados contextos mais complexos, como interrupções no sistema. Um exemplo interessante é o estudo de Ramshani et al. (2019), que aborda um modelo de localização de dois níveis (Two-Level UFLP) sob incertezas de disrupção. Os autores desenvolvem formulações matemáticas e heurísticas para lidar com paralisações em pontos da rede, demonstrando como essas descontinuidades afetam a seleção de locais e as alocações. Esses avanços ilustram a tendência atual de modelagem mais realista dos problemas de localização, incorporando aspectos de resiliência operacional e logística associada às instalações.

\section{Fundamentação Teórica}

Os problemas de localização de instalações envolvem a decisão de designar uma ou mais unidades para atender a um determinado contexto. Como a formulação matemática não admite soluções fracionárias, por exemplo, “meia instalação”, tais problemas são tradicionalmente modelados por meio da Programação Inteira, em que as variáveis assumem valores inteiros e representam decisões binárias de abertura ou não de uma instalação.

Segundo Owen e Daskin (1998), embora a Programação Inteira seja a base predominante das formulações clássicas, essas abordagens costumam ser estáticas e determinísticas, pois não consideram explicitamente as incertezas do ambiente real, o que limita sua aplicabilidade em contextos dinâmicos ou sujeitos a variações imprevistas. Nessas circunstâncias, embora matematicamente consistentes, as soluções obtidas podem não refletir integralmente a complexidade prática do problema.

Como alternativas, Owen e Daskin (1998) sugerem métodos dinâmicos e estocásticos. Os métodos dinâmicos são indicados quando o cenário presente é conhecido, mas há interesse em projetar soluções para períodos futuros, considerando possíveis alterações no sistema. Já os métodos estocásticos permitem modelar explicitamente a incerteza, ao incorporar variáveis aleatórias relacionadas a fatores como demanda, custos e tempos de viagem. Nesse caso, a solução do problema busca identificar a localização ótima das instalações por meio de avaliações probabilísticas das variáveis ou da análise de cenários alternativos, caracterizando um modelo de planejamento baseado em cenários.

Neste trabalho, serão desenvolvidas análises fundamentadas em duas formulações clássicas da literatura: o Problema das P-Medianas, formulado por Hakimi (1964), e o Problema de Localização de Máxima Cobertura (PLMC), descrito por Church e ReVelle (1976). O Problema das P-Medianas consiste em alocar instalações de modo a minimizar o tempo médio de deslocamento entre cada cliente e a instalação que o atende. Nesse contexto, a formulação busca identificar posições ideais de instalações de forma que a acessibilidade seja maximizada e os custos de deslocamento sejam reduzidos ao mínimo possível. A formulação matemática do problema das P-Medianas pode ser expressa como um modelo de Programação Inteira da seguinte forma:
\[
\begin{aligned}
i \in I\ (I=\{1,\ldots,n\}) &\quad \text{índice/conjunto de clientes},\\
j \in J\ (J=\{1,\ldots,m\}) &\quad \text{índice/conjunto de instalações potenciais},\\
c_{ij} \ge 0 &\quad \text{custo/distância entre o cliente $i$ e a instalação $j$},\\
P \in \mathbb{Z}_{+} &\quad \text{número de instalações a serem abertas}.
\end{aligned}
\]
\[
x_{ij} =
\begin{cases}
1, & \text{se o cliente $i$ é atendido pela instalação $j$},\\
0, & \text{caso contrário},
\end{cases}
\quad
y_{j} =
\begin{cases}
1, & \text{se a instalação $j$ é aberta},\\
0, & \text{caso contrário}.
\end{cases}
\]
\begin{alignat*}{2}
\text{Minimizar}\quad 
& \sum_{i \in I}\sum_{j \in J} c_{ij}\,x_{ij} \\
\text{sujeito a}\quad
& \sum_{j \in J} x_{ij} = 1,                 && \quad \forall i \in I, \\
& \sum_{j \in J} y_{j} = P,                  && \\
& x_{ij} \le y_{j},                          && \quad \forall i \in I,\ \forall j \in J, \\
& x_{ij} \in \{0,1\},                        && \quad \forall i \in I,\ \forall j \in J, \\
& y_{j} \in \{0,1\},                         && \quad \forall j \in J.
\end{alignat*}

O PLMC, por sua vez, busca localizar um número limitado de instalações para maximizar a cobertura da demanda em uma distância máxima \(d\). Diferentemente do problema das P-Medianas, que objetiva minimizar as distâncias médias de deslocamento, o PLMC considera que um cliente está coberto se estiver a uma distância aceitável \(d\) de pelo menos uma instalação. Assim, a formulação prioriza atender o maior número possível de clientes (ou demanda) dentro de um raio \(d\), dado um número \(P\) de instalações a serem abertas.

\[
\begin{aligned}
w_i \ge 0 &\quad \text{demanda do cliente } i,\\
d_{ij} \ge 0 &\quad \text{distância entre o cliente $i$ e a instalação $j$},\\
P \in \mathbb{Z}_{+} &\quad \text{número de instalações a serem abertas},\\
d \ge 0 &\quad \text{distância máxima de cobertura}.
\end{aligned}
\]
Definem-se as variáveis:
\[
z_i = 
\begin{cases}
1, & \text{se o cliente $i$ está coberto por alguma instalação}, \\
0, & \text{caso contrário},
\end{cases}
\quad
x_j =
\begin{cases}
1, & \text{se a instalação $j$ é aberta}, \\
0, & \text{caso contrário}.
\end{cases}
\]
O modelo de Programação Inteira do PLMC pode ser formulado como:
\begin{alignat*}{2}
\text{Maximizar}\quad
& \sum_{i \in I} w_i\,z_i \\
\text{sujeito a}\quad
& \sum_{j \in J} x_j = P, \\
& z_i \le \sum_{j : d_{ij} \le d} x_j, && \quad \forall i \in I, \\
& x_j \in \{0,1\}, && \quad \forall j \in J, \\
& z_i \in \{0,1\}, && \quad \forall i \in I.
\end{alignat*}

No cenário descrito a seguir, esses dois métodos foram utilizados para analisar a disposição de instalações de farmácias na região central de Aracaju, com base em dados obtidos pelo OpenStreetMap. Os bairros foram representados em uma projeção cartográfica e os estabelecimentos foram identificados a partir dos dados do OpenStreetMap. Em seguida, avaliou-se o posicionamento dessas instalações segundo o problema das P-Medianas e o PLMC, conforme detalhado a seguir.

\section{Metodologia}

A metodologia adotada neste trabalho tem como objetivo demonstrar, de forma aplicada, o uso de dois modelos clássicos de Localização de Instalações: o Problema das P-Medianas e o Problema da Máxima Cobertura Limitada por $p$ Instalações (PLMC ou MCLP). Para isso, estruturou-se um procedimento composto por três etapas: (i) extração e preparação dos dados do cenário; (ii) modelagem conceitual dos nós de demanda e de instalação; e (iii) implementação e execução computacional das formulações.

\subsection{Análise do Cenário}

Esta seção apresenta, de forma integrada, os procedimentos analíticos adotados para examinar o recorte urbano selecionado, bem como os resultados parciais obtidos durante o processo. Orientada por uma abordagem exploratória, a análise fundamenta-se em representações cartográficas, medidas simplificadas de acessibilidade e conceitos dos modelos de P-Medianas e do Problema de Localização de Máxima Cobertura (PLMC).

Adotou-se um enfoque predominantemente visual e qualitativo, utilizando mapas temáticos para observar a relação entre pontos de demanda e a disponibilidade de serviços farmacêuticos. Para fins desta análise, o ``atendimento'' de um cliente assume caráter ilustrativo: nas P-Medianas é interpretado como distância média ao estabelecimento mais próximo, enquanto no PLMC corresponde ao raio de cobertura definido para cada instalação.

A análise foi organizada em três etapas principais: (i) coleta e preparação dos dados; (ii) modelagem conceitual dos problemas; e (iii) construção das matrizes utilizadas nos modelos.

\subsubsection{Coleta e Preparação dos Dados}

Os dados espaciais foram obtidos a partir da base do OpenStreetMaps (OSM), utilizando-se duas ferramentas complementares: a plataforma uMap, empregada para estilização e visualização do mapa, e a API OverPass, responsável pela consulta dinâmica e geração dos arquivos GeoJSON.

Foram identificadas as farmácias localizadas nos bairros Cirurgia, Centro, Getúlio Vargas e São José, na região central de Aracaju. Para a definição dos nós de demanda, selecionaram-se elementos urbanos associados a fluxo de pessoas ou concentração de serviços potenciais, filtrados pelas seguintes \emph{tags} do OSM: 
\begin{itemize}
    \item \textbf{Nós de demanda}: \textit{school}, \textit{university}, \textit{hospital}, \textit{clinic}, \textit{doctors}, \textit{social\_facility}, \textit{community\_centre}, \textit{bus\_station}, \textit{public\_transport}, \textit{stop\_position}, e \textit{platform}.
    \item \textbf{Instalações potenciais}: \textit{pharmacy}.
\end{itemize}

Ressalta-se que os resultados obtidos refletem apenas o conteúdo presente na base do OSM, sem verificação em campo, não se pode garantir a completude ou exatidão das informações.

\begin{figure}[H]
    \centering
    \includegraphics[width=0.70\linewidth]{Artigo/map-nos-osm.png}
    \caption{Nós de demanda marcados em vermelho, e nós de serviço existentes marcados em cinza.}
    \vspace{-10pt}
    \label{fig:pmedian}
\end{figure}

\subsubsection{Modelagem Conceitual dos Problemas}

Em virtude das limitações computacionais da ferramenta utilizada na simulação, procedeu-se a uma redução do conjunto inicial de pontos, mantendo-se apenas os grupos mais representativos de demanda e algumas ocorrências isoladas. Após o refinamento, obtiveram-se 14 nós de cliente e 10 nós de possíveis instalações.

\begin{figure}[H]
    \centering
    \includegraphics[width=0.70\linewidth]{Artigo/nos-filtrados.PNG}
    \caption{Conjunto reduzido de nós utilizados nas simulações.}
    \vspace{-8pt}
    \label{fig:nos-filtrados}
\end{figure}

Com essa configuração, a plataforma uMap foi utilizada para extrair manualmente as distâncias necessárias à formulação do problema de P-Medianas, gerando, para cada instalação, a distância correspondente a todos os nós de cliente. Os valores coletados foram organizados em uma matriz de distâncias. Para adequação aos limites do simulador desenvolvido, os valores foram posteriormente escalonados por um fator de $1/10$.

As Tabela~\ref{tab:distancias-parte1} e Tabela~\ref{tab:distancias-parte2} apresentam a matriz de distâncias resultante.

\begin{table}[H]
    \centering
    \caption{Matriz de distâncias (Parte 1): Clientes C1--C7}
    \label{tab:distancias-parte1}
    \begin{tabular}{lccccccc}
        \hline
            & C1 & C2 & C3 & C4 & C5 & C6 & C7 \\
        \hline
            F1  & 1.3km & 1.8km & 2.3km & 2km & 1.9km & 2.1km & 2.2km \\
            F2  & 1km & 336.70m & 374.81m & 406.97m & 933.93m & 1.4km & 1.8km \\
            F3  & 1.2km & 660.42m & 559.01m & 168.71m & 551.85m & 1.1km & 1.7km \\
            F4  & 1.3km & 1km & 1.1km & 677.66m & 169.22m & 523.92m & 971.11m \\
            F5  & 1.6km & 1.2km & 1.4km & 835.89m & 313.40m & 431.22m & 686.60m \\
            F6  & 1.2km & 1.1km & 1.5km & 1.1km & 787.35m & 1km & 1.2km \\
            F7  & 1.8km & 1.8km & 2.1km & 1.6km & 1.2km & 1.2km & 1.4km \\
            F8  & 2.1km & 1.9km & 2km & 1.4km & 942.38m & 647.22m & 619.98m \\
            F9  & 2.2km & 1.8km & 1.9km & 1.4km & 724.15m & 543.20m & 349.11m \\
            F10 & 2.5km & 2km & 2km & 1.4km & 841.65m & 544.91m & 92.10m \\
        \hline
    \end{tabular}
\end{table}

\begin{table}[H]
    \centering
    \caption{Matriz de distâncias (Parte 2): Clientes C8--C14}
    \label{tab:distancias-parte2}
    \begin{tabular}{lccccccc}
        \hline
            & C8 & C9 & C10 & C11 & C12 & C13 & C14 \\
        \hline
            F1  & 846.18m & 1.4km & 1.5km & 928.27m & 1.3km & 800.51m & 363.16m \\
            F2  & 1.2km & 1.1km & 1.4km & 1.6km & 1.8km & 1.7km & 2.1km \\
            F3  & 1.1km & 916.40m & 1km & 1.3km & 1.5km & 1.7km & 2km \\
            F4  & 820.42m & 486.63m & 529.46m & 973.95m & 869.65m & 1.8km & 1.9km \\
            F5  & 818.83m & 477.83m & 315.57m & 718.90m & 740.03m & 1.8km & 2km \\
            F6  & 245.88m & 145.89m & 535.46m & 374.30m & 729.50m & 1.3km & 1.3km \\
            F7  & 670.99m & 712.27m & 700.72m & 221.14m & 509.07m & 1.6km & 1.3km \\
            F8  & 1.1km & 788.22m & 434.65m & 819.25m & 370.32m & 2.2km & 2.1km \\
            F9  & 1.7km & 1.2km & 851.18m & 1.3km & 922.80m & 2.7km & 2.6km \\
            F10 & 1.3km & 914.80m & 530.57m & 1km & 550.94m & 2.4km & 2.3km \\
        \hline
    \end{tabular}
\end{table}

\subsection{Implementação da Ferramenta}

Com o intuito de facilitar a exploração dos modelos, desenvolveu-se uma aplicação web demonstrativa, não orientada à execução de algoritmos de otimização em larga escala. A aplicação foi implementada como uma \textit{Single Page Application} utilizando React (via Next.js) e TypeScript, com gerenciamento de estado via Zustand. A arquitetura privilegia simplicidade, modularidade e possibilidade de futuras integrações com solvers mais robustos.

\subsubsection{Métodos de Solução Desenvolvidos}

Foram implementados dois solucionadores enumerativos, ambos exaustivos:

\begin{itemize}
    \item \textbf{P-Medianas}: enumeração de todas as combinações possíveis de $p$ instalações, utilizando como entrada a matriz de custos cliente–instalação.
    \item \textbf{PLMC/MCLP}: enumeração de todas as combinações de $p$ instalações visando maximizar o número de clientes cobertos, com base na matriz binária de cobertura.
\end{itemize}

\subsubsection{Construção da Matriz de Custos}

A matriz de custos do modelo de P-Medianas é gerada a partir do conjunto de instalações possíveis. Cada instalação contém um vetor de demandas indexado por cliente, permitindo construir uma matriz em que cada linha representa um cliente e cada coluna corresponde a uma instalação candidata. O código responsável por essa operação é:

\begin{verbatim}
const buildCostMatrix = (facilities) => {
    const clientCount = valid[0].demand.length;
    return Array.from({ length: clientCount }, 
            (_, clientIdx) =>
                facilities.map(f => f.demand[clientIdx].cost)
    );
};
\end{verbatim}

A matriz gerada pode ser visualizada conceitualmente como:

\[
\begin{bmatrix}
\text{cost}_{1,1} & \text{cost}_{1,2} & \cdots & \text{cost}_{1,m} \\
\text{cost}_{2,1} & \text{cost}_{2,2} & \cdots & \text{cost}_{2,m} \\
\vdots            & \vdots            & \ddots & \vdots            \\
\text{cost}_{n,1} & \text{cost}_{n,2} & \cdots & \text{cost}_{n,m}
\end{bmatrix}
\]

onde cada elemento \(\text{cost}_{i,j}\) corresponde ao custo obtido diretamente pelo código acima, por meio de \texttt{f.demand[clientIdx].cost}.

\subsubsection{Avaliação das Combinações de Instalações}

Após construída a matriz de custos, o modelo avalia todas as combinações possíveis de instalações de tamanho \(p\). Cada combinação representa uma solução candidata. Para cada solução, o algoritmo percorre todos os clientes e determina o menor custo de alocação entre as instalações pertencentes ao conjunto analisado. O cálculo é implementado da seguinte forma:

\begin{verbatim}
for (const combo of getCombinations(facilityIds, p)) {
    let totalCost = 0;
    for (let c = 0; c < nClients; c++) {
        let minCost = Infinity;
        for (const f of combo) {
            if (costMatrix[c][f] < minCost) 
                minCost = costMatrix[c][f];
        }
        totalCost += minCost;
    }
}
\end{verbatim}

Esse procedimento representa, em termos computacionais, a operação de:

\[
\text{totalCost} = 
\sum_{c=1}^{nClients} 
\min(\text{costMatrix}[c][f_1],\,
     \text{costMatrix}[c][f_2],\,
     \ldots,\,
     \text{costMatrix}[c][f_p]),
\]

mas substituindo a fórmula matemática pelo cálculo explícito presente no código.

Assim, cada cliente é associado à instalação mais barata dentre as selecionadas, e a soma desses menores custos define a qualidade da solução. O algoritmo avalia todas as combinações possíveis e identifica aquela cujo valor acumulado de \texttt{totalCost} é o menor entre todas as alternativas.

\subsubsection{Matriz de Cobertura do PLMC}

No PLMC, a matriz de cobertura é representada por uma estrutura binária que indica se cada cliente está dentro do raio de cobertura \(R\) de cada instalação candidata. Cada linha da matriz corresponde a um cliente e cada coluna representa uma instalação. O valor é igual a \(1\) quando o cliente está coberto pela instalação e \(0\) caso contrário. O cálculo é realizado diretamente a partir das coordenadas dos pontos:

\begin{verbatim}
const coverageMatrix = (facilities, demands, radius) => {
    return demands.map(d =>
        facilities.map(f => {
            const dist = Math.sqrt((d.posX - f.posX) ** 2 
                + (d.posY - f.posY) ** 2);
            return dist <= radius ? 1 : 0;
        })
    );
};
\end{verbatim}

Desse modo, a estrutura resultante pode ser representada conceitualmente como:

\[
\begin{bmatrix}
b_{1,1} & b_{1,2} & \cdots & b_{1,m} \\
b_{2,1} & b_{2,2} & \cdots & b_{2,m} \\
\vdots  & \vdots  & \ddots & \vdots  \\
b_{n,1} & b_{n,2} & \cdots & b_{n,m}
\end{bmatrix},
\]

\subsubsection{Avaliação da Cobertura no PLMC}

Após construída a matriz de cobertura, o PLMC seleciona $p$ instalações e avalia quantos clientes são atendidos por ao menos uma das instalações escolhidas. Esse processo envolve analisar todas as combinações possíveis de tamanho $p$ e medir, para cada uma delas, o alcance total obtido. Para cada combinação, o algoritmo verifica, cliente a cliente, se alguma das instalações da combinação o cobre, usando diretamente os valores da matriz binária.

\begin{verbatim}
for (const combo of getCombinations(facilityIds, p)) {
    let covered = 0;
    for (let d = 0; d < nDemands; d++) {
        if (combo.some(f => coverageMatrix[d][f] === 1)) {
            covered++;
        }
    }
}
\end{verbatim}

Essa lógica corresponde à operação:

\[
\text{covered} = 
\sum_{d=1}^{nDemands}
\mathbf{1}\left(\exists\, f \in \text{combo} \;\;|\;\;
\text{coverageMatrix}[d][f] = 1\right),
\]

mas, em vez de empregar equações formais, utiliza diretamente o teste computacional:

\texttt{combo.some(f => coverageMatrix[d][f] == 1)}.

Dessa forma, o número total de nós atendidos é contabilizado explicitamente, e a melhor solução é aquela cuja combinação maximiza o valor acumulado de \texttt{covered}.

\section{Resultados}

Como resultado do procedimento adotado, foram produzidos tanto uma aplicação funcional, capaz de simular pequenas instâncias dos problemas de P-Medianas e PLMC, quanto a resolução referente ao cenário apresentado anteriormente. A seguir, descreve-se separadamente a aplicação desenvolvida e os resultados obtidos na análise.

\subsection{Aplicação Produzida}

A aplicação possibilita ao usuário selecionar entre duas formulações clássicas de localização de P-Medianas e PLMC. Em ambas as formulações, o valor de \(p\) é informado diretamente no cabeçalho da interface. Em seguida, o usuário pode inserir seus nós de cliente e de instalação utilizando o botão \texttt{Add +} disponível em cada lista. Cada nó recebe um nome definido pelo próprio usuário e pode ser removido clicando no ícone \texttt{X} correspondente.

No modelo de P-Medianas, após a criação dos nós, o usuário seleciona uma instalação e fornece o custo de associação dessa instalação a todos os clientes. Os campos de entrada são exibidos automaticamente abaixo dos ícones dos clientes. A simulação somente pode ser executada quando todos os custos estiverem informados. Uma vez concluído o preenchimento, basta acionar o botão \texttt{Solve} para que, se não houver inconsistências nos dados, a solução seja exibida em um painel disposto na parte inferior da página.

Para o PLMC, o usuário informa o raio de cobertura diretamente no cabeçalho. Os nós podem ser posicionados livremente em um \textit{canvas} interativo localizado na metade inferior da interface. Diferentemente da formulação de P-Medianas, não é necessário posicionar todos os nós para realizar a simulação. Após a execução, as instalações não selecionadas são exibidas em vermelho, enquanto a instalação escolhida é conectada aos clientes que se encontram dentro de seu raio de cobertura.

A aplicação encontra-se disponível em: \texttt{https://e-graph.vercel.app}.

\subsection{Resultado da Análise}

Com os dados preparados, o simulador foi utilizado para resolver as formulações de P-Medianas e PLMC, cada uma com limite de instalações \(P = 1\). O objetivo dessa configuração foi evidenciar, em cada modelo, a instalação que melhor atende aos critérios da formulação.

\begin{figure}[H]
\centering
\includegraphics[width=1\linewidth]{Artigo/resolucao_pmedianas.jpg}
\caption{Solução para o Problema das P-Medianas.}
\vspace{-10pt}
\label{fig:pmedian}
\end{figure}

Para o Problema das \(p\)-Medianas, a solução indica que o nó de instalação F6 apresenta o menor custo agregado de distância em relação aos nós de demanda, totalizando aproximadamente 12,53 km. Dessa forma, conforme a formulação adotada, F6 é a instalação mais eficiente para minimizar a distância média entre os clientes e o serviço.

\begin{figure}[H]
\centering
\includegraphics[width=1\linewidth]{Artigo/resolucao_plmc.jpg}
\caption{Solução para o PLMC.}
\vspace{-10pt}
\label{fig:plmc}
\end{figure}

No caso do PLMC, com \(P = 1\) e raio de cobertura igual a 1,25 km, o simulador selecionou a instalação F5, que cobre quatro clientes. Ainda assim, observa-se que a cobertura obtida é limitada, evidenciando que uma única instalação não é suficiente para atender de forma satisfatória à distribuição espacial da demanda no cenário analisado.

\section{Conclusão}

Os resultados obtidos demonstram soluções compatíveis com o comportamento esperado para cada formulação. No caso das P-Medianas, a instalação F6 foi selecionada por ocupar uma posição relativamente central no conjunto de nós de demanda, favorecendo a minimização das distâncias médias em um cenário caracterizado por dispersão espacial. Já na formulação do PLMC, a instalação F5 foi indicada por estar mais próxima do agrupamento principal de demanda, maximizando o número de clientes atendidos dentro do raio estipulado, embora a cobertura resultante permaneça limitada diante da configuração espacial analisada.

Essas conclusões são preliminares e devem ser interpretadas à luz das restrições metodológicas adotadas, tais como:

\begin{itemize}
\item O raio de cobertura estabelecido para o PLMC, que, apesar de funcional para fins demonstrativos, é arbitrário e não considera fatores como mobilidade urbana, infraestrutura viária ou diferenças entre modais de deslocamento;
\item O caráter essencialmente exploratório da análise, conduzida sem otimizações formais, calibração de parâmetros ou modelagem estocástica da demanda, o que restringe a robustez e a generalização dos resultados.
\end{itemize}

Recomenda-se que trabalhos futuros ampliem o escopo analítico e didático voltado aos Problemas de Localização de Instalações. Os achados indicam que mesmo as formulações mais simples desse campo são capazes de gerar perspectivas intuitivas e coerentes com a realidade espacial estudada, tornando-se valiosas para fins de ensino e compreensão conceitual. Nesse sentido, futuras investigações podem considerar:

\begin{itemize}
\item A incorporação de serviços de GIS à camada de visualização, permitindo análises interativas, reprodutíveis e georreferenciadas com maior precisão;
\item A expansão da aplicação para modelos dinâmicos ou estocásticos de Problemas de Localização de Instalações, aprimorando a representação da demanda ao longo do tempo.
\end{itemize}

Em síntese, apesar das limitações, o estudo evidencia o potencial dos modelos clássicos de localização de instalações quando aplicados a dados abertos do OSM, oferecendo um panorama inicial das relações espaciais entre serviços e potenciais usuários no recorte urbano estudado.

\begin{thebibliography}{99}

\bibitem{SousaFilho2012} FILHO, G. S. et al. Uma arquitetura e ferramentas para problemas de localização de facilidades no setor público. São Paulo: SBC, 2012. Disponível em: <https://sol.sbc.org.br/index.php/sbsi/article/view/14428>

\bibitem{Rushton1979} RUSHTON, G. Optimal Location of Facilities. [s.l.] Department of Geography, University of Iowa, jan. 1979.

\bibitem{KuoKung2025} KUNG, L.-C.; CHUANG, J.-S.; KUO, Y.-T. Optimal Allocation of Capacitated Facilities considering Time-dependent User Preference for User Number Maximization. Disponível em: <https://ssrn.com/abstract=4276750>. Acesso em: 5 set. 2025. 

\bibitem{Busing2024} BÜSING, C.; GERSING, T.; WREDE, S. Insights into the computational complexity of the single-source capacitated facility location problem with customer preferences. [s.l.] Optimization Online, dez. 2024. 

\bibitem{Kang2023} KANG, C.-N. et al. A service facility location problem considering customer preference and facility capacity. Computers \& Industrial Engineering, v. 177, p. 109070, 2023.

\bibitem{OwenDaskin1998} OWEN, S. H.; DASKIN, M. S. Strategic facility location: A review. European Journal of Operational Research, v. 111, p. 423–447, 1998. 

\bibitem{Alarcon-GerbierBuscher2022} ALARCONGERBIER, E.; BUSCHER, U. Modular and mobile facility location problems: A systematic review. Computers \& Industrial Engineering, v. 173, p. 108734, 2022. 

\bibitem{ChurchReVelle2010} CHURCH, R.; REVELLE, C. Theoretical and Computational Links between the pMedian, Location Setcovering, and the Maximal Covering Location Problem. Geographical Analysis, v. 8, p. 406–415, 3 set. 2010. 

\bibitem{Pedroso2012} PEDROSO, J. P. et al. Facility Location Problems — Mathematical Optimization: Solving Problems Using Gurobi and Python. Disponível em: <https://scipbook.readthedocs.io/en/latest/flp.html>.

\end{thebibliography}

\end{document}