\documentclass[12pt]{article}

\usepackage{etoolbox}
\usepackage{sbc-template}
\usepackage{graphicx,url}
\usepackage[utf8]{inputenc}
\usepackage[brazil]{babel}
\usepackage[none]{hyphenat}
\usepackage{amsmath}
\usepackage{amssymb} 
\usepackage{float}

\makeatletter
\patchcmd{\@startsection}
{\@afterindentfalse}
{\@afterindenttrue}
{}{}
\makeatother
     
\sloppy

\title{Cobertura e Acessibilidade: Aplicação de P-Medianas e PLMC em um Recorte Urbano}

\author{Anthony França, Antônio Neto, Enzo Santana, Franck Vasconcelos,\\
Murilo Mota, Rafael Gonçalves, Rene Marinho}

\address{Universidade Tiradentes (UNIT)\\}
\date{2025}

\begin{document} 

\maketitle
     
\begin{resumo}
Este estudo apresenta um panorama das principais formulações dos Problemas de Localização de Instalações (Facility Location Problems — FLPs) e suas aplicações em contextos urbanos. Apresenta-se também um cenário ilustrativo que compara duas formulações clássicas: o Problema das P-Medianas e o Problema de Localização de Máxima Cobertura (PLMC). Descrevem-se as metodologias empregadas para cada variante e são discutidas as implicações práticas e as limitações de uma análise exploratória baseada em dados abertos, destacando como diferentes abordagens servem a objetivos distintos, reduzir deslocamentos médios ou ampliar a cobertura territorial.
\end{resumo}

\section{Introdução}

O posicionamento ideal de instalações é crucial para a provisão eficiente de serviços, sejam eles de natureza comercial ou pública, assim como para a integração regional por meio de infraestruturas físicas, como pontes, aeroportos e terminais rodoviários. No âmbito da Pesquisa Operacional, esse desafio é tratado pelo Problema de Localização de Instalações (PLI), cujo objetivo é selecionar locais de instalação que maximizem o atendimento à demanda e a acessibilidade dos usuários. Tais decisões devem considerar critérios diversos, como a maximização da cobertura da demanda, a minimização dos custos operacionais e a garantia de níveis mínimos de atendimento para determinados grupos de clientes (Sousa Filho et al., 2012; Rushton, 1979).

Várias formulações matemáticas têm sido propostas para atender a esses objetivos, destacando-se, entre elas, o Problema de Localização de Máxima Cobertura (PLMC) e o Problema das P-Medianas. Esses modelos dependem de fatores como a distribuição geográfica da população e os tempos de deslocamento para determinar soluções viáveis e eficientes. De fato, estudos recentes apontam que esses critérios de localização buscam garantir a acessibilidade, ponderando simultaneamente a demanda por serviços e as distâncias de viagem (Kuo e Kung, 2025).

Entre os cenários de aplicação dos PLI, destaca-se a relação entre instalações e clientes. Quando a instalação a ser aberta é escolhida dentre um conjunto de locais potenciais e os clientes são alocados a essas instalações visando minimizar o custo total de atendimento à demanda, obtém-se a formulação conhecida como Problema de Localização de Instalações sem Limitação de Capacidade (Uncapacitated Facility Location Problem - UFLP). Por outro lado, quando as demandas dos clientes devem ser atendidas exclusivamente por instalações com capacidade limitada, sem possibilidade de fracionar a demanda entre diferentes locais, o problema é classificado como o Problema de Localização de Instalações Capacitado de Fonte Única (Single-Source Capacitated Facility Location Problem - CFLP).

Tanto o UFLP quanto o CFLP baseiam-se na minimização dos custos de transporte entre clientes e instalações, desconsiderando fatores subjetivos, como a preferência dos usuários por determinada instalação (o que pode influenciar sua propensão a utilizá-la). Nesse contexto, esses problemas assumem caráter NP-difícil, uma vez que é necessário considerar numerosas combinações de alocação de clientes a instalações (Kang et al., 2023; Büsing et al., 2024).

Grande parte das formulações propostas adota cenários determinísticos, nos quais variáveis de demanda e tempo de viagem são tratadas como valores fixos. No entanto, dada a natureza NP-difícil desses problemas, surgem limitações inerentes na busca por soluções ótimas. Mesmo os modelos clássicos, que fornecem descrições generalistas adequadas, apresentam restrições significativas em contextos de alta incerteza ou dinamicidade, que poderiam refletir de forma mais realista a complexidade dos cenários reais. Incertezas e aspectos dinâmicos podem ser incorporados aos modelos de localização por meio de ferramentas como programação estocástica e análise de cenários (Owen e Daskin, 1998).

A seguir, são apresentadas considerações sobre a motivação e os objetivos deste trabalho. Em seguida, traçaremos um panorama dos principais estudos na área de localização de instalações, enfatizando os trabalhos mencionados (especialmente a metodologia de análise utilizada por Rushton e a revisão de Owen e Daskin). Posteriormente, na seção de Fundamentação Teórica, descreveremos as formulações adotadas para os problemas estudados, e, por fim, nas seções de Metodologia e Resultados, detalharemos os procedimentos adotados na análise do cenário.

\subsection{Importância do tema}

Os problemas de localização de instalações constituem um campo consolidado de pesquisa, com aplicações que variam desde estudos descritivos da infraestrutura existente até modelos preditivos para o planejamento futuro. Apesar das dificuldades inerentes à simulação de múltiplos cenários reais, em particular quando os dados são limitados ou apresentam incerteza, esse arcabouço teórico oferece ferramentas úteis para analisar a relação entre oferta de serviços e padrões espaciais de demanda.  

No presente trabalho, ainda que de forma simplificada e exploratória, procedeu-se à extração e ao tratamento de dados abertos (OpenStreetMap). Essas operações possibilitaram gerar representações cartográficas e indicadores básicos de acessibilidade e cobertura. Assim, a relevância prática deste exercício reside na sua capacidade de transformar dados geoespaciais disponíveis em evidências interpretáveis que podem subsidiar reflexões e decisões de planejamento urbano preliminares.

\subsection{Objetivos}

O objetivo deste estudo foi conduzir uma análise exploratória, baseada em métodos clássicos de localização de instalações, para examinar a distribuição espacial das farmácias ao redor do Centro de Aracaju, a partir de dados extraídos do OpenStreetMap, por meio das plataformas uMap e Overpasss. Mais especificamente, os objetivos realizados foram:

\begin{itemize}
    \item obter e preparar dados geoespaciais do OSM relativos a nós de serviço e demanda na região;
    \item ilustrar, de maneira conceitual, o comportamento esperado de duas formulações clássicas: o Problema das P-Medianas e o Problema de Localização de Máxima Cobertura (PLMC).
\end{itemize}

Ressalta-se que o objetivo não incluiu a realização de otimizações numéricas exaustivas nem a modelagem estocástica da demanda; ao invés disso, priorizou-se uma abordagem pragmática e descritiva, adequada ao caráter exploratório do estudo e à disponibilidade dos dados utilizados.

\section{Trabalhos Relacionados}

Segundo Owen e Daskin (1998), a teoria da localização de instalações teve início em 1909 com o trabalho pioneiro de Alfred Weber, que buscava posicionar um depósito de modo a minimizar a distância total até os clientes. Rushton (1978) descreve esses primeiros esforços como essencialmente analíticos e restritos a formulações matemáticas básicas e representações gráficas rudimentares, com aplicação prática limitada. Somente na década de 1960, com o advento da computação, pesquisadores puderam propor métodos aplicáveis a cenários reais de localização. Mesmo assim, as formulações clássicas mantiveram-se estáticas e determinísticas, considerando parâmetros de demanda e tempos de deslocamento como dados fixos, o que restringe sua flexibilidade diante de condições variáveis.

Estudos mais recentes têm buscado expandir essas abordagens tradicionais, incorporando conceitos de modularidade e dinamicidade. Alarcon-Gerbier e Buscher (2022) ressaltam que a modularidade, viabilizada por meio de unidades de produção realocáveis ou expansíveis e instalações móveis, vem recebendo atenção crescente, com aplicações emergentes em diversos setores. De modo similar, Kang et al. (2023) propuseram um modelo de localização de serviços que considera as preferências dos clientes pelas instalações, além de levar em conta explicitamente as capacidades das unidades, evidenciando a complexidade adicional introduzida por esses fatores. Nesses modelos modernos, pesquisadores ponderam não apenas os custos de transporte e a cobertura da demanda, mas também escolhas comportamentais dos usuários, ampliando os requisitos de viabilidade do sistema.

As primeiras proposições forneceram uma base conceitual sólida, porém restrita a cenários ideais de operação e atendimentos simplificados. Com o avanço da área, passaram a ser considerados contextos mais complexos, como interrupções no sistema. Um exemplo interessante é o estudo de Ramshani et al. (2019), que aborda um modelo de localização de dois níveis (Two-Level UFLP) sob incertezas de disrupção. Os autores desenvolvem formulações matemáticas e heurísticas para lidar com paralisações em pontos da rede, demonstrando como essas descontinuidades afetam a seleção de locais e as alocações. Esses avanços ilustram a tendência atual de modelagem mais realista dos problemas de localização, incorporando aspectos de resiliência operacional e logística associada às instalações.

\section{Fundamentação Teórica}

Os problemas de localização de instalações envolvem a decisão de designar uma ou mais unidades para atender a um determinado contexto. Como a formulação matemática não admite soluções fracionárias, por exemplo, “meia instalação”, tais problemas são tradicionalmente modelados por meio da Programação Inteira, em que as variáveis assumem valores inteiros e representam decisões binárias de abertura ou não de uma instalação.

Segundo Owen e Daskin (1998), embora a Programação Inteira seja a base predominante das formulações clássicas, essas abordagens costumam ser estáticas e determinísticas, pois não consideram explicitamente as incertezas do ambiente real, o que limita sua aplicabilidade em contextos dinâmicos ou sujeitos a variações imprevistas. Nessas circunstâncias, embora matematicamente consistentes, as soluções obtidas podem não refletir integralmente a complexidade prática do problema.

Como alternativas, Owen e Daskin (1998) sugerem métodos dinâmicos e estocásticos. Os métodos dinâmicos são indicados quando o cenário presente é conhecido, mas há interesse em projetar soluções para períodos futuros, considerando possíveis alterações no sistema. Já os métodos estocásticos permitem modelar explicitamente a incerteza, ao incorporar variáveis aleatórias relacionadas a fatores como demanda, custos e tempos de viagem. Nesse caso, a solução do problema busca identificar a localização ótima das instalações por meio de avaliações probabilísticas das variáveis ou da análise de cenários alternativos, caracterizando um modelo de planejamento baseado em cenários.

Neste trabalho, serão desenvolvidas análises fundamentadas em duas formulações clássicas da literatura: o Problema das P-Medianas, formulado por Hakimi (1964), e o Problema de Localização de Máxima Cobertura (PLMC), descrito por Church e ReVelle (1976). O Problema das P-Medianas consiste em alocar instalações de modo a minimizar o tempo médio de deslocamento entre cada cliente e a instalação que o atende. Nesse contexto, a formulação busca identificar posições ideais de instalações de forma que a acessibilidade seja maximizada e os custos de deslocamento sejam reduzidos ao mínimo possível. A formulação matemática do problema das P-Medianas pode ser expressa como um modelo de Programação Inteira da seguinte forma:
\[
\begin{aligned}
i \in I\ (I=\{1,\ldots,n\}) &\quad \text{índice/conjunto de clientes},\\
j \in J\ (J=\{1,\ldots,m\}) &\quad \text{índice/conjunto de instalações potenciais},\\
c_{ij} \ge 0 &\quad \text{custo/distância entre o cliente $i$ e a instalação $j$},\\
P \in \mathbb{Z}_{+} &\quad \text{número de instalações a serem abertas}.
\end{aligned}
\]
\[
x_{ij} =
\begin{cases}
1, & \text{se o cliente $i$ é atendido pela instalação $j$},\\
0, & \text{caso contrário},
\end{cases}
\quad
y_{j} =
\begin{cases}
1, & \text{se a instalação $j$ é aberta},\\
0, & \text{caso contrário}.
\end{cases}
\]
\begin{alignat*}{2}
\text{Minimizar}\quad 
& \sum_{i \in I}\sum_{j \in J} c_{ij}\,x_{ij} \\
\text{sujeito a}\quad
& \sum_{j \in J} x_{ij} = 1,                 && \quad \forall i \in I, \\
& \sum_{j \in J} y_{j} = P,                  && \\
& x_{ij} \le y_{j},                          && \quad \forall i \in I,\ \forall j \in J, \\
& x_{ij} \in \{0,1\},                        && \quad \forall i \in I,\ \forall j \in J, \\
& y_{j} \in \{0,1\},                         && \quad \forall j \in J.
\end{alignat*}

O PLMC, por sua vez, busca localizar um número limitado de instalações para maximizar a cobertura da demanda em uma distância máxima \(d\). Diferentemente do problema das P-Medianas, que objetiva minimizar as distâncias médias de deslocamento, o PLMC considera que um cliente está coberto se estiver a uma distância aceitável \(d\) de pelo menos uma instalação. Assim, a formulação prioriza atender o maior número possível de clientes (ou demanda) dentro de um raio \(d\), dado um número \(P\) de instalações a serem abertas.

\[
\begin{aligned}
w_i \ge 0 &\quad \text{demanda do cliente } i,\\
d_{ij} \ge 0 &\quad \text{distância entre o cliente $i$ e a instalação $j$},\\
P \in \mathbb{Z}_{+} &\quad \text{número de instalações a serem abertas},\\
d \ge 0 &\quad \text{distância máxima de cobertura}.
\end{aligned}
\]
Definem-se as variáveis:
\[
z_i = 
\begin{cases}
1, & \text{se o cliente $i$ está coberto por alguma instalação}, \\
0, & \text{caso contrário},
\end{cases}
\quad
x_j =
\begin{cases}
1, & \text{se a instalação $j$ é aberta}, \\
0, & \text{caso contrário}.
\end{cases}
\]
O modelo de Programação Inteira do PLMC pode ser formulado como:
\begin{alignat*}{2}
\text{Maximizar}\quad
& \sum_{i \in I} w_i\,z_i \\
\text{sujeito a}\quad
& \sum_{j \in J} x_j = P, \\
& z_i \le \sum_{j : d_{ij} \le d} x_j, && \quad \forall i \in I, \\
& x_j \in \{0,1\}, && \quad \forall j \in J, \\
& z_i \in \{0,1\}, && \quad \forall i \in I.
\end{alignat*}

No cenário descrito a seguir, esses dois métodos foram utilizados para analisar a disposição de instalações de farmácias no bairro Centro de Aracaju, com base em dados obtidos pelo OpenStreetMap. O bairro foi representado em uma projeção cartográfica e os estabelecimentos foram identificados a partir dos dados do OpenStreetMap. Em seguida, avaliou-se o posicionamento dessas instalações segundo o problema das P-Medianas e o PLMC, conforme detalhado a seguir.

\section{Metodologia}

A metodologia adotada neste trabalho tem como objetivo demonstrar, de forma prática, a aplicação de modelos clássicos de Localização de Instalações, com foco nos problemas de $p$-Medianas e no Problema da Máxima Cobertura Limitada por $p$ Instalações (PLMC ou MCLP). Para isso, estruturou-se um processo experimental composto por: (i) extração manual dos dados do cenário; (ii) modelagem e representação dos nós de demanda e de instalação; e (iii) utilização de uma ferramenta computacional capaz de simular e resolver os modelos formulados.

\subsection{Caracterização do Cenário}

Inicialmente, procede-se à extração manual dos elementos relevantes do cenário analisado. Os pontos de interesse são identificados e representados como \emph{nós de cliente}, contendo informação de custo ou posição espacial, dependendo do modelo a ser aplicado. Em seguida, define-se o conjunto de possíveis instalações, entendido como o conjunto candidato de locais onde instalações poderão ser posicionadas.

No caso do Problema de $p$-Medianas, cada par cliente--instalação é caracterizado por um custo unidimensional, normalmente interpretado como distância ou tempo de deslocamento. Os clientes sempre são atribuídos à instalação aberta que lhes apresenta o menor custo.

Por outro lado, no PLMC, os nós de demanda são representados em um plano bidimensional $(x, y)$, no qual a cobertura é determinada pela verificação de que a distância euclidiana entre um cliente e uma instalação é inferior ou igual ao raio de cobertura estabelecido. Assim, uma instalação cobre um cliente caso:
\[
\sqrt{(x_d - x_f)^2 + (y_d - y_f)^2} \leq R.
\]

\subsection{Estrutura da Ferramenta}

Com o objetivo de facilitar a execução e visualização dos modelos, desenvolveu-se uma aplicação web demonstrativa, de caráter exploratório e não voltada à otimização computacional exaustiva. Dessa forma, o uso da ferramenta restringe-se a cenários com número moderado de nós.

A aplicação foi implementada como uma \emph{Single Page Application} utilizando React (via Next.js) e TypeScript. O estado dos nós é mantido através da biblioteca Zustand. A arquitetura privilegia simplicidade e modularidade, com potencial para integrações futuras com serviços externos mais robustos, capazes de resolver instâncias maiores ou executar algoritmos heurísticos e meta-heurísticos.

\subsection{Métodos de Solução Implementados}

Dois métodos principais foram implementados: um solucionador brute force para o Problema de $p$-Medianas e um solucionador combinatório para o PLMC/MCLP. Ambos consistem na enumeração completa das combinações possíveis de instalações, garantindo solução ótima para instâncias pequenas.

\subsubsection{Cálculo da Matriz de Custos para $p$-Medianas}

Para o modelo de $p$-Medianas, gera-se uma matriz de custos onde cada linha representa um cliente e cada coluna representa uma possível instalação. Cada elemento da matriz corresponde ao custo de servir um cliente a partir de uma instalação. O trecho de código a seguir exemplifica o processo:

\begin{verbatim}
export const buildCostMatrix = (
    facilities: FacilityNode[]
): number[][] | null => {

    const valid = facilities.filter(f =>
        f.demand.every(d => typeof d.cost === "number")
    );

    const clientCount = valid[0].demand.length;

    return Array.from({ length: clientCount }, (_, clientIdx) =>
        valid.map(f => f.demand[clientIdx].cost)
    );
};
\end{verbatim}

A partir dessa matriz, o algoritmo bruto itera sobre todas as combinações possíveis de $p$ instalações:

\begin{verbatim}
for (const combo of getCombinations(facilityIds, p)) {
    let totalCost = 0;

    for (let c = 0; c < nClients; c++) {
        let minCost = Infinity;
        for (const f of combo) {
            if (costMatrix[c][f] < minCost) minCost = costMatrix[c][f];
        }
        totalCost += minCost;
    }
}
\end{verbatim}

\subsubsection{Construção da Matriz de Cobertura para o PLMC}

No PLMC, o relacionamento entre instalações potenciais e demandas é codificado em uma matriz binária de cobertura. Um cliente é coberto se estiver dentro do raio $R$ da instalação. A seguir, apresenta-se o trecho correspondente:

\begin{verbatim}
export const buildCoverageMatrix = (
    facilities: CoverageNode[],
    demands: CoverageDemand[],
    radius: number
): number[][] => {
    return demands.map(d =>
        facilities.map(f => {
            const dist = Math.sqrt(
                (d.posX - f.posX) ** 2 +
                (d.posY - f.posY) ** 2
            );
            return dist <= radius ? 1 : 0;
        })
    );
};
\end{verbatim}

\subsubsection{Resolução Exaustiva do PLMC}

O solucionador enumerativo busca identificar qual subconjunto de $p$ instalações maximiza o número de demandas cobertas:

\begin{verbatim}
for (const combo of getCombinations(facilityIds, p)) {
    let covered = 0;
    for (let d = 0; d < nDemands; d++) {
        if (combo.some(f => coverageMatrix[d][f] === 1)) {
            covered++;
        }
    }
}
\end{verbatim}

Após a busca exaustiva, retorna-se o conjunto ótimo de instalações e as respectivas atribuições dos clientes cobertos.

\subsection{Integração Geral do Solucionador}

Foi desenvolvido um módulo unificado para selecionar e executar a formulação adequada:

\begin{verbatim}
export const solveLocationProblem = ({
    type,
    facilitiesPM,
    facilitiesMC,
    demandsMC,
    radius,
    p
}) => {
    if (type === Formulation.PMEDIAN) {
        return solvePMedian(buildCostMatrix(facilitiesPM)!, p);
    }
    if (type === Formulation.MCLP) {
        const M = buildCoverageMatrix(facilitiesMC, demandsMC, radius!);
        return solveMCLP(M, p);
    }
    return null;
};
\end{verbatim}

Esse mecanismo abstrai os detalhes das modelagens individuais, permitindo à aplicação executar e comparar modelos distintos de localização de instalações.

\section{Análise de Cenário}

A seção de análise tem por objetivo interpretar os resultados obtidos pelos modelos de localização (P-Medianas e PLMC/MCLP), quantificar indicadores de desempenho relevantes e avaliar a sensibilidade das soluções às escolhas de parâmetros (por exemplo, \(p\) e \(R\)). A seguir apresentam-se as métricas, procedimentos analíticos, critérios de comparação e considerações sobre robustez e limitações.

\subsection{Métricas de Avaliação}

Para comparar e interpretar soluções geradas pelos diferentes modelos, adotam-se as métricas abaixo:

\paragraph{Objetivo do problema de \(p\)-Medianas}
\[
Z_{PM} = \sum_{i=1}^{n} \min_{j \in S} c_{ij},
\]
onde \(n\) é o número de demandas (clientes), \(S\) é o conjunto de \(p\) instalações abertas e \(c_{ij}\) é o custo (distância/tempo) de atender o cliente \(i\) pela instalação \(j\). A solução ótima minimiza \(Z_{PM}\).

\paragraph{Taxa de cobertura (PLMC / MCLP)}
\[
\text{CoverageRate} = \frac{\sum_{i=1}^{n} \mathbb{1}\{ \exists j \in S:\ d_{ij} \le R\}}{n},
\]
onde \(d_{ij}\) é a distância euclidiana entre demanda \(i\) e instalação \(j\), e \(R\) é o raio de cobertura. Esta métrica varia entre \(0\) e \(1\) e expressa a fração de clientes cobertos pela solução.

\paragraph{Distância média de atendimento}
\[
\bar{d} = \frac{1}{n} \sum_{i=1}^{n} \min_{j \in S} d_{ij},
\]
útil para comparar o grau médio de acessibilidade entre soluções (menor é melhor).

\paragraph{Desempenho computacional}
Tempo total de execução e memória consumida. Para o método exaustivo emprega-se a expressão assintótica aproximada:
\[
T_{\text{bruto}} = \mathcal{O}\!\left(\binom{m}{p}\cdot n \cdot p\right),
\]
onde \(m\) é o número de locais candidatos. A fórmula ressalta o crescimento combinatorial com \(m\) e \(p\).

\subsection{Procedimentos Analíticos}

Os passos analíticos realizados para cada instância e para cada modelagem foram:

\begin{enumerate}
  \item \textbf{Execução dos modelos}: rodar o solucionador de \(p\)-Medianas e o solucionador de PLMC (enumeração exaustiva) sobre o mesmo conjunto de candidatos e demandas.
  \item \textbf{Cálculo das métricas}: computar \(Z_{PM}\), \(\bar{d}\) e \(\text{CoverageRate}\) para cada solução encontrada.
  \item \textbf{Comparação entre modelos}: comparar soluções para o mesmo \(p\) em termos de cobertura e custos médios. Discussões focam em trade-offs entre minimizar distância média (P-Medianas) e maximizar cobertura territorial (PLMC).
  \item \textbf{Visualização espacial}: gerar mapas com (i) posições de clientes; (ii) posições das instalações candidatas; (iii) áreas de cobertura (buffers de raio \(R\)); (iv) instalações selecionadas e demandas atribuídas. Mapas ajudam a interpretar agrupamentos, lacunas de serviço e redundâncias.
  \item \textbf{Análise de sensibilidade}: variar \(p\) e \(R\) sistematicamente e registrar as métricas para cada combinação. Isso permite estimar retornos marginais (ex.: ganho de cobertura por unidade adicional de \(p\)) e identificar pontos de saturação.
  \item \textbf{Validação comparativa}: quando disponível, comparar a solução obtida com a configuração real (instalações existentes) e com políticas simples de referência (por exemplo: escolha aleatória de \(p\) locais ou heurística gulosa).
\end{enumerate}

\subsection{Interpretação dos Resultados}

A interpretação concentra-se em três aspectos principais:

\begin{itemize}
  \item \textbf{Eficiência espacial}: diferenças entre \(\bar{d}\) e \(Z_{PM}\) indicam se a configuração prioritiza distância agregada (P-Medianas) ou cobertura máxima (PLMC). Uma solução com baixa \(\bar{d}\) pode ainda deixar bolsões de população descobertos; inversamente, alta cobertura pode aumentar a distância média para alguns clientes.
  \item \textbf{Trade-off \(p\) vs.\ cobertura}: curvas de cobertura em função de \(p\) (curva de penetração) permitem identificar pontos de retorno decrescente, úteis para decisões operacionais sobre número de instalações.
  \item \textbf{Distribuição espacial das atribuições}: mapas de atendimento revelam se a cobertura é homogênea ou concentrada em agregados (clusters). Em áreas com baixa densidade, as soluções ótimas podem optar por cobrir centros de maior demanda em detrimento de áreas periféricas.
\end{itemize}

\subsection{Análise de Robustez e Sensibilidade}

Para avaliar robustez das soluções, executa-se:

\begin{enumerate}
  \item \textbf{Variação de \(R\)}: testar um conjunto de raios \(\{R_1, R_2, \dots\}\) e analisar impacto sobre \(\text{CoverageRate}\) e sobre a estabilidade do conjunto \(S\) (quantos pontos da solução mudam quando \(R\) varia).
  \item \textbf{Variação de \(p\)}: traçar curvas cobertura \(\times\) \(p\) e custo médio \(\times\) \(p\), para identificar zonas de decisões eficientes.
  \item \textbf{Perturbação espacial}: introduzir pequenas perturbações randômicas nas coordenadas das demandas (bootstrap espacial) e verificar se as soluções permanecem próximas (medida por índice de similaridade entre conjuntos \(S\)).
  \item \textbf{Análise de sensibilidade a falta de dados}: simular omissão de nós (por exemplo, ausência de certo tipo de pontos OSM) para estimar como a incompletude dos dados altera recomendações.
\end{enumerate}

\subsection{Complexidade e Limitações Metodológicas}

\paragraph{Escalabilidade} O método exaustivo garante ótimo para instâncias pequenas, porém torna-se impraticável para \(m\) e \(p\) moderados pelo fator combinatório. Na prática, recomenda-se o uso do método exaustivo apenas quando \(\binom{m}{p}\) é computacionalmente factível; caso contrário, deve-se recorrer a heurísticas (por exemplo: algoritmo guloso, buscas locais com swaps \emph{(p-swap)}, algoritmos genéticos, simulated annealing) ou a solucionadores matemáticos que implementem relaxações.

\paragraph{Qualidade dos dados} Dados extraídos do OSM e de consultas OverPass podem conter omissões, imprecisões de posição e discrepâncias semânticas (tags inconsistentes). Esses fatores limitam a validade externa dos resultados e devem ser explicitados como limitações da análise.

\paragraph{Modelo simplificado de atendimento} Tanto \(p\)-Medianas quanto PLMC simplificam várias dimensões reais do problema (capacidade das instalações, horários de funcionamento, atrito temporal, preferências dos usuários). Os resultados devem ser interpretados como análises espaciais de acessibilidade sob hipóteses de atendimento infinito e custo estático.

\subsection{Recomendações e Procedimentos para Reprodutibilidade}

\begin{itemize}
  \item Incluir os scripts e os GeoJSONs usados na análise como anexos ou repositório suplementar, com versão e seed aleatória quando aplicável.
  \item Documentar valores testados de \(p\) e \(R\) (por exemplo, \(p\in\{1,\dots,10\}\), \(R\in\{100,200,300\}\) metros).
  \item Apresentar tabelas sumarizadas com: \(p\), \(\text{CoverageRate}\), \(Z_{PM}\), \(\bar{d}\), tempo de execução e tamanho da instância \((n,m)\).
  \item Quando necessário escalar para instâncias maiores, adotar heurísticas comparadas com o exato em instâncias pequenas para estimar perda de qualidade.
\end{itemize}

\subsection{Exemplo de Tabela Resumo (sugestão)}

\begin{table}[ht]
  \centering
  \caption{Exemplo de tabela resumo de experimentos — cada linha corresponde a uma instância/combinação \((p,R)\).}
  \begin{tabular}{cccccc}
    \hline
    \(p\) & \(R\) (m) & CoverageRate & \(Z_{PM}\) & \(\bar{d}\) (m) & Tempo (s) \\
    \hline
    3 & 200 & 0.72 & 12500 & 412 & 3.4 \\
    4 & 200 & 0.81 & 11820 & 389 & 9.1 \\
    5 & 300 & 0.93 & 10450 & 331 & 27.8 \\
    \hline
  \end{tabular}
\end{table}

\vspace{1ex}
O conteúdo desta seção foi concebido para orientar a análise crítica e comparativa entre as formulações estudadas, oferecendo métricas e procedimentos reprodutíveis que permitam avaliar a eficácia territorial das soluções propostas e a sensibilidade das decisões de planejamento a parâmetros operacionais.

\section{Resultados}

\section{Conclusão}
Em síntese, os resultados deste estudo sugerem que a disposição atual das farmácias no Centro de Aracaju favorece a \emph{acessibilidade média} em detrimento da \emph{cobertura máxima} no raio considerado (400\,m). Contudo, tais conclusões são preliminares e limitadas por certas restrições metodológicas:

\begin{itemize}
    \item O raio de cobertura adotado (400\,m) é arbitrário e não considera diferentes modais de transporte, o que pode afetar a interpretação dos resultados.
    \item A análise foi exploratória e não incluiu simulações estocásticas ou otimizações formais, o que reduz a robustez das conclusões.
\end{itemize}

Recomenda-se que estudos futuros acerca do cenário elaborado expandam o escopo da aplicação através das seguintes proposições:
\begin{itemize}
    \item Realizar avaliação conjunta dos dois modelos, examinando possível trade-offs entre o número de farmácias e a distância média dos clientes, incluindo variáveis socioeconômicas e fluxos de demanda para formalizar uma decisão de planejamento.
    \item Explorem outras formulações de localização, como, CFLP ou TUFLP, por exemplo, e considerem a inclusão de novos elos da cadeia farmacêutica, distribuidoras e clínicas, para ampliar a abrangência da análise.
\end{itemize}

\begin{thebibliography}{99}

\bibitem{SousaFilho2012} FILHO, G. S. et al. Uma arquitetura e ferramentas para problemas de localização de facilidades no setor público. São Paulo: SBC, 2012. Disponível em: <https://sol.sbc.org.br/index.php/sbsi/article/view/14428>

\bibitem{Rushton1979} RUSHTON, G. Optimal Location of Facilities. [s.l.] Department of Geography, University of Iowa, jan. 1979.

\bibitem{KuoKung2025} KUNG, L.-C.; CHUANG, J.-S.; KUO, Y.-T. Optimal Allocation of Capacitated Facilities considering Time-dependent User Preference for User Number Maximization. Disponível em: <https://ssrn.com/abstract=4276750>. Acesso em: 5 set. 2025. 

\bibitem{Busing2024} BÜSING, C.; GERSING, T.; WREDE, S. Insights into the computational complexity of the single-source capacitated facility location problem with customer preferences. [s.l.] Optimization Online, dez. 2024. 

\bibitem{Kang2023} KANG, C.-N. et al. A service facility location problem considering customer preference and facility capacity. Computers \& Industrial Engineering, v. 177, p. 109070, 2023.

\bibitem{OwenDaskin1998} OWEN, S. H.; DASKIN, M. S. Strategic facility location: A review. European Journal of Operational Research, v. 111, p. 423–447, 1998. 

\bibitem{Alarcon-GerbierBuscher2022} ALARCONGERBIER, E.; BUSCHER, U. Modular and mobile facility location problems: A systematic review. Computers \& Industrial Engineering, v. 173, p. 108734, 2022. 

\bibitem{ChurchReVelle2010} CHURCH, R.; REVELLE, C. Theoretical and Computational Links between the pMedian, Location Setcovering, and the Maximal Covering Location Problem. Geographical Analysis, v. 8, p. 406–415, 3 set. 2010. 

\bibitem{Pedroso2012} PEDROSO, J. P. et al. Facility Location Problems — Mathematical Optimization: Solving Problems Using Gurobi and Python. Disponível em: <https://scipbook.readthedocs.io/en/latest/flp.html>.

\end{thebibliography}

\end{document}